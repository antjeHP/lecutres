\section{Rechnerarithmetik}
\begin{example}
Betrachte $F=F(10,5,-4,5)$ und Maschinenzahlen
\begin{align*}
x=2.5684 \cdot 10^{0}= 2.56840000 \\
y=3.2791 \cdot 10^{-3} = 0.0032791
\end{align*}
Es gilt:\\
\begin{center}
$\begin{rcases}
	x+y= 2.5716792 \\
	x-y= 2.5651209 \\
	x \cdot y= 0.00842204044 \\
	\frac{x}{y}= 783.2637004
\end{rcases} \not\in F$
\end{center}
\end{example}
\begin{remark}
Die Menge $F(b,t,e_{min},e_{max})$ ist nicht abgeschlossen bezüglich der Grundrechenarten und können somit im Allgemeinen nicht im Computer implementiert werden.
\end{remark}
\paragraph{Lösung}
Wir runden das Ergebnis und implementieren so eine Pseudoarithmetik.
Das bedeutet, wir ersetzen $\circ \in  \{+,-,\cdot, \div\}$ durch $\boxcircle \in \{\boxplus, \boxminus, \boxdot, \boxslash\}$ definiert durch:
\begin{equation}\label{eqn:rundung}
x \boxcircle y \coloneqq rd (x \circ y)
\end{equation}
Auf Hardwareebene wird üblicherweise mit einer längeren Matisse gearbeitet und dann normalisiert und gerundet. Dies entspricht dem IEEE 754 Standard.
\begin{remark}
Für $|x|,|y|, |x \circ y| \in [z_{min},z_{max}]$ impliziert \eqref{eqn:rundung}, dass
\[
	\frac{|x \boxcircle y - x \circ y|}{|x \circ y|}= \frac{rd(x \circ y - x \circ y)}{|x \circ y|} \le \varepsilon_{mach}
\]

Das bedeutet, dass $\boxcircle$ im Computer bestmöglich umgesetzt ist.
\end{remark}
\begin{example}
Betrachte $F=F(10,5,-4,5)$ \\
\begin{itemize}
	\item Setze:
\begin{itemize}
	\item $a=0.98765$
	\item $b=0.012424$
	\item $c=-0.0065432$
\end{itemize}
Dann gilt:
\[
	(a+b)+c=a+(b+c)=0.9925208
\]
Numerisch gilt:
\[
(0.98765 \boxplus 0.012424)\boxminus 0.0065432= rd(0.9935568)=0.99356
\]
und
\[
0.98765 \boxplus (0.012424\boxminus 0.0065432)=rd(0.9935308)=0.99353
\]
\item Setze
	\begin{itemize}
		\item $a=4.2832$
		\item $b=-4.2821$
		\item $5.7632$
	\end{itemize}
Dann gilt
\[
(a+b)\cdot c = a \cdot c + b \cdot c = 0.006339520000001
\]
Numerisch gilt:
\end{itemize}
\end{example}
\begin{remark}
Mathematisch äquivalente Algorithmen auf Fließkommazahlen können je nach Implementierung zu wesentlich unterschiedlichen Ergebnissen führen, selbst wenn die Eingangszahlen exakt dargestellt werden.
\end{remark}
\subsection{Auslöschung}
Unglücklicherweise pflanzen sich numerische Fehler, zum Beispiel durch Rundung im Verlauf eines Algorithmus fort.

\begin{lemma}[Fehlerfortpflanzung]
Es seien $x,y \in \R$ mit Datenfehlern $\Delta x, \Delta y \in  \R$ behaftet, welche\[
|\frac{\Delta x}{x}| , |\frac{\Delta y}{y}| \ll 1
\]
erfüllen. Für $\circ \in  \{+,-,\cdot, \div\}$ gilt für den fortgepflanzten Fehler
\[
\Delta (x \circ y) \coloneqq (x+ \Delta x) \circ (y + \Delta y) - x \circ y \text{, }
\]
dass
\begin{equation*}
\begin{split}
\frac{\Delta (x \pm y)}{x \pm y} = \frac{x}{x \pm y}\frac{\Delta x}{x} \pm \frac{y}{x \pm y} \frac{\Delta y}{y} \\
\frac{\Delta (x \cdot y)}{xy} \approx \frac{\Delta x}{x}+ \frac{\Delta y}{y} \\
\frac{\Delta (\frac{x}{y})}{\frac{x}{y}} \approx \frac{\Delta x}{x}-\frac{\Delta y}{y}
\end{split}
\end{equation*}
Hierbei bedeutet "$\approx$", dass Terme mit $(\Delta x)^2, (\Delta y)^2, \Delta x \Delta y$ vernachlässigt werden.
\end{lemma}
\begin{proof}
Übung.
\end{proof}
Für $\cdot,\div$ addieren bzw. subtrahieren sich die Fehler.
\paragraph{Achtung:} Ist $|x \pm y|$ wesentlich kleiner als $|x|$ oder $|y|$, kann der relative Fehler massiv verstärkt werden. 
Dieses Phänomen heißt Auslöschung.\\
Bei der Konstruktion von Algorithmen sollte Auslöschung möglichst vermieden werden.
\begin{example}\label{eg:pq}
Betrachte \\
\begin{equation}
\label{eqn:pq}
x^2-2px+q=0
\end{equation}
mit Lösungen:
\[
x_{1,2} = p \pm \sqrt[2]{p^2-q} 
\]


\begin{algorithm}[H]
 \label{alg:nst1}
 \caption{naive Nullstellenberechung}
 \KwData{$p,q \in \R$ so ,dass \eqref{eqn:pq} lösbar ist}
 \KwResult{Nullstellen $x_1,x_2$ von \eqref{eqn:pq}.}
 $d=\sqrt{p\cdot p -q}$ \\
 $x_1= p+d$ \\
 $x_2= p-d$
 \end{algorithm}
 Mit $p=100$, $q=1$ in dreistelliger, dezimaler Gleitkommaarithmetik ergibt sich:
 \begin{equation*}
	 \begin{split}
		 d= \sqrt{10000-1}= \sqrt{rd(9999)}=\sqrt{10000} = 100 \\
		 x_1=100+100=200 \\
		 x_2= 100-100 =0
	 \end{split}
 \end{equation*}
Die exakten Werte sind  $x_1 \approx 199.99, x_2 \approx 0.00500$, welche in dreistelliger, dezimaler Gleitkommaarithmetik als $x_1=200,x_2=0$ dargestellt werden. \\
Gemäß \ref{thm:emach} ist die Maschinengenauigkeit:
\[
\varepsilon_{mach} = \frac{1}{2}b^{1-t}= \frac{1}{2}10^{1-3}=0.005
\]
Der relative Fehler $\frac{|0-0.005|}{|0.005|}=1$ ist also zu 100 Prozent falsch.
\end{example}
Wir können die Genauigkeit von $x_2$ erhöhen, indem wir den Wurzelsatz von Vieta $x_1x_2=q$ verwenden.\\
\begin{algorithm}[H]
 \caption{verbesserte Nullstellenberechnung}
 \KwData{$p,q \in  \R$ so ,dass \eqref{eqn:pq} lösbar ist.}
 \KwResult{Nullstellen $x_1,x_2$ \eqref{eqn:pq}.}
 $d= \sqrt[2]{p \cdot p -q}$ \\
 \If{$q \ge 0$}{$x_1=p+d$
 \Else{$x_1=p-d$}}
 $x_2=\frac{q}{x_1}$
 \end{algorithm}
 Wir erhalten nun $d=100, x_1=200,x_2=\frac{1}{200}=0.005$
