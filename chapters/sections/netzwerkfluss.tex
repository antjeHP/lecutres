\section{Netzwerkflussprobleme}
\begin{example}
Ausgehend von einer Bananenplantage s sollen alle geernteten Bananen zum Lagerhaus t transportiert werden. Für den Transport stehen Straßen mit $r_1,\ldots,r_p$ $\frac{kg}{n}$ Transportkapazität zu den Seehäfen $A_1,\ldots,A_p$ zu Verfügung. Von den Zielhäfen $B_1,\ldots,B_q$ stehen Transportkapazitäten von $d_1,\ldots,d_q$ $\frac{kg}{n}$ zum Supermarkt t bereit.
Die Transportkapazität zwischen den Seehäfen werden mit $c(A_i,A_j), 1\le i\le p, 1\le j \le p$ bezeichnet.
\paragraph{Fragen:}
\begin{itemize}
	\item Ist es mögliche, alle Transportkapazitäten auszuschöpfen?
	\item Falls nein, was ist die maximal mögliche Transportkapazität?
	\item Wie sollen die Bananenladungen aufgeteilt werden?
\end{itemize}
Konstruiere einen gewichteten Digraph $G=(V,E,w)$ mit
\begin{itemize}
	\item $V=\{s, A_1,\ldots,A_p, B_1,\ldots,q,t\} $ 
	\item $E= \{(s,A_i), (A_i,B_j), (B_j,t) ; 1\le i\le p, 1\le j\le p\} $ 
	\item $w(e)= \begin{cases}
			r_i &, e=(s,A_i) \\
			c(A_i,B_j) &, e(A_i,B_j) \\
			d_i &, e(B_i,t)
	\end{cases}$ 
\end{itemize}

\begin{center}
\begin{tikzpicture}


	%Netzwerk einfügen


\end{tikzpicture}
\end{center}
\end{example}
\begin{definition}
Ein \emph{Netzwerk} ist ein Tupel $N=(V,E,c,s,t)$  bestehend aus
\begin{itemize}
	\item einem gewichteten Digraphen $G=(V,E,c)$
	\item einer \emph{Kapazitätsfunktion} $c \colon E \to R_{\ge 0}$
	\item einer Quelle $s \in V$ mit $pre(s)= \emptyset$ 
	\item einer Senke $t \in V$ mit $post(t)=\emptyset$ 
\end{itemize}
Ein Fluss $f \colon E \to R_{\ge 0}$ ist eine Funktion, die folgende Bedingungen erfüllt:
\begin{enumerate}
	\item \emph{Kapazitätsbedingung:}
		\[
		f(v,w) \le c(v,w)
		\]
	\item \emph{Kirchhoffsches Gesetz:}
		\[
		\sum_{u \in pre(v)} f(u,v) = \sum_{w \in  post(v)} f(v,w)
		\]
für alle $v \in V \setminus \{s,t\} $ .
\end{enumerate}
Der \emph{Wert} des Flusses ist:
\[
flow(f) = \sum_{w \in post(s)}f(s,w)
\]
Der \emph{maximale FLuss} von N wird bezeichnet als:
\[
MaxFlow(N) = \max \{flow(f) \text{ ist Fluss für N}\} 
\]
Eine Flussfunktion $f$ wird \emph{optimal} genannt, falls 
\[
flow(f)=MaxFlow(N)
\]
ist. \\
Ein \emph{Schnitt} für N ist eine Knotenmenge $S \subset V$ mit $s \in S, t \not\in S$. \\
Die \emph{Kapazität} eines Schnittes ist gegeben durch:
\[
cap(S)= \sum_{v \in S \\ w \in post(v) \setminus S}c(v,w)
\]
Die minimale Schnittkapazität von N ist:
\[
MinCut(N)= \min \{cap(S) | \text{ S ist Schnitt für N}\} 
\]
\end{definition}
\begin{lemma}
	Sei S ein Schnitt für $N=(V,E,c,s,t)$. Dann gilt für jeden Fluss f, dass
	\begin{align*}
		flow(f) &= \sum_{w \in post(v) \setminus S} f(v,w) - \sum_{u \in  pre(v)} f(u,v) \\
		flow(f) &\le cap(S)
	\end{align*}
\end{lemma}
\begin{proof}
Rechne:
\begin{align*}
	flow(f) &= \sum_{w \in post(s)} f(s,w) \\
		&= \sum_{v \in S}\left( \sum_{w \in post(v)}f(v,w)- \sum_{u \in pre(v)}f(u,v) \right)\\
		&=\sum_{w \in post(v)}f(v,w) - \sum_{u \in pre(v) \setminus S}f(u,v)
\end{align*}
Für die nächste Behauptung können wir ebenfalls nachrechnen:
\begin{align*}
	flow(f) &= \sum_{w \in post(v) \setminus S} f(v,w) - \sum_{u \in pre(v) \setminus S}f(u,v) \\
		&\le \sum_{w \in post(v) \setminus S}c(v,w) \\
		&=cap(S)
\end{align*}
\end{proof}
