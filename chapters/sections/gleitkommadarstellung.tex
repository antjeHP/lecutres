\section{Gleitkommadarstellung}
\begin{definition}[Gleitkommadarstellung]
	Die Gleitkommastellung einer Zahl $z \in  \R$ ist gegeben durch:
	\[
	z=+-m\cdot b^{e}
	\]
mit einer \underline{Matisse} m, dem \underline{Exponenten} e und der \underline{Basis} b.
\end{definition}
Der Einfachheit halber nehmen wir an, dass die Basis für alle Zahlen gleich ist, obwohl sie prinzipiell verschieden sein könnte.
\begin{example}
Die Gleitkommadarstellungen von:
\begin{itemize}
	\item $(-384.753)_{10}= -3.84753 \cdot 10^2$
	\item $(0.00042)_{10}=4.2 \cdot 10^{-4d}$
	\item $(1010.101)_2= (1.010101)_2 \cdot 2^{3}$
\end{itemize}
\end{example}
\paragraph{Achtung:} Die Gleitkommadarstellung ist \underline{nicht} eindeutig!

\begin{definition}
Die Matisse m heißt \underline{normalisiert}, falls $m=m_1.m_2m_3 \ldots m_t$ mit $1\le m_1 \le b$.
\end{definition}
Um die eindeutige, normalisierte Gleitkommadarstellung im Computer zu speichern, müssen wir festlegen, wie viele Stellen für Matisse und Exponent zur Verfügung gestellt werden. Dies resultiert in der Menge 
\[
F=F(b,t,e_{min}, e_{max}) = \{z= +-m_1.m_2m_3\ldots m_t \cdot b^{e} | e_{min} \le e \le e_{max} \} 
\]
Da 0 nicht in dieser Form dargestellt werden kann, reserviert man dafür eine spezielle Ziffernfolge in Matisse und Exponent.
%Hier fehlt wieder ein schöne Grafik
\begin{remark}
Das Hidden Bit ist hilfreich
\begin{itemize}
	\item Für $b=2$ kann die erste Ziffer $m_1$ der Matisse weggelassen werden (Hidden Bit)
	\item Um den Exponenten besser vergleichen zu können, verwendet man oft die Exzess- oder Bias-Darstellung. Durch Addition der Exzesser $|e_{min}|+1$ wird der Exponetn auf den Bereich $1,2,\ldots, |e_{min}|+e_{max} +1 $ transformiert.
\end{itemize}
\end{remark}
\begin{example}[IEEE 754 Standard]
	Betrachte binäre Gleitkommazhalen mit 64 Bit auf dem Computer. 
	\begin{itemize}
		\item 52 Bit für die Matisse in Hidden Bit Darstellung
		\item 11 Bit für den Exponenten mit $e_{min} = -1022$ und $e_{max} =1023$ gespeichert in Exzessdarstellung.
		\item 1 Bit als Vorzeichen.
	\end{itemize}
Weiter Fälle können Online nachgelesen werden.
\end{example}

\subsubsection{Genauigkeit der Gleitkommadarstellung}
Da die Menge aller Zahlen in $F=F(b,t,e_{min},e_{max})$ endlich ist, müssen wir Zahlen in $\R \setminus  F$ geeignet annähern.

\begin{definition}
Die Rundung ist eine Abbildung $rd\colon \R \to F $ mit 
\begin{itemize}
	\item $rd(a)=a$, $a \in F$
	\item $rd(z)$, $z \in  \R$ ist gegeben so, dass $|z-rd(z)|=\underset{a \in F}{\min}(z-a)$ für alle $z \in  \R$.
\end{itemize}
\end{definition}
\begin{definition}[Maschinengenauigkeit]
Den maximalen relativen Rundungsfehler $\varepsilon_{mach}$ für $z_{min} \le |z| \le z_{max}$ nennt man Maschinengenauigkeit.
Die Stellen der Mantisse heißen \underline{signfikante Stellen}.
Wenn die Mantisse t-stellig ist, spricht man von \underline{t-stelliger Arithmetik}.
\end{definition}
\begin{theorem}[Maschinengenauigkeit von F]\label{thm:emach}
	Die Maschinengenauigkeit $\varepsilon_{mach}$ für $F=F(b,t,e_{min},e_{max})$ ist:
	\[
	\varepsilon_{mach} = \frac{1}{2} b^{1-t} \text{.}
	\]
\end{theorem}
\begin{proof}
Betrachte relativen Rundungsfehler $\varepsilon$, der vom Abscheiden nicht signifikanten Stellen herrüht. Sei $z_{min} < x < z_{max}$ und $\tilde{x}$ die durch Abschneiden entstehende Zahl.
Dann gilt:
\begin{align*}
	\varepsilon
	&= \frac{|x-\tilde{x}|}{|x|} \\
	&= \frac{|x_1.x_2x_3 \ldots x_t x_{t+1}\ldots \cdot b^{e}-x_1x_2x_3\ldots x_t \cdot b^{e}}{|x|} \\
	&=\frac{|0.x_{t+1}\ldots| \cdot |b^{e+1-t}|}{|x|}
\end{align*}
Da $|0.x_{t+1}\ldots|<1$ und $b^{e}\le |x|$ gilt:
\[
\varepsilon <\frac{b^{e+1-t}}{b^{e}}=b^{1-t}
\]
Da der Rundungsfehler $\varepsilon_{mach}$ höchsten halb so groß ist wie der Abschneidefehler folgt die Behauptung.
\end{proof}
Die Maschinengenauigkeit $\varepsilon_{mach}$ ist die wichtigste Größe zur Beurteilung der Genauigkeit von Gleitkommarechnungen am Computer. Sie gibt Aufschluss zur Anzahl signifikanter Stellen zur Basis $b$.
Die zugehörige Anzahl $s$ Stellen im Dezimalsystem erhält man durch das Auflösen von:
\[
\varepsilon_{mach} = \frac{1}{2}b^{1-t}=\frac{1}{2}10^{1-s}
\]
nach s. Es folgt daher:
\[
s= \lfloor 1+(t-1)\log_{10}(b) \rfloor
\]
Für den IEEE 754 Standard mit $b=2, t=53$ erhält man $s=16$ signifikante Stellen.
