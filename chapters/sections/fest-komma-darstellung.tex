\section{Fest-Komma-Darstellung}
\begin{definition}
	Bei der Festkommadarstellung einer n-stelligen Zahl werden k Vorkomma und n-k Nachkommastellen definiert:
	\[
	z=+-(z_{k-1} \ldots z_0.z_{-1} \ldots z_{k-n})_b = +- \sum_{i=k-n}^{k-1}z_i b^{i}
	\]

\end{definition}
\begin{example}
Im Zehner System wie gehabt.\\
Im Binärsystem ergibt sich folgendes: $(101.01)_2= 2^{2}+ 2^{0}+ 2^{-2}=5.25$
\end{example}
\paragraph{Achtung} Im Gegensatz zur Darstellung ganzer Zahlen können bereits bei der Konvertierung von Dezimalzahlen in das b-adische Zahlensystem Rundungsfehler auftreten.
\begin{example}
	$(0.8)_{10}=(0.110\overline{1100})_2$
\end{example}
Das größte Problem der Festkommadarstellung ist alllerdings, dass der darstellbare Bereich stark eingeschränkt ist und schlecht aufgelöst ist, da der Abstand zwischen zweier Zahlen immer gleich ist.
\begin{example}
Die kleinste darstellbare Zahl in Fixkommadarstellung ist \[
z_1=(0 \ldots 0.0 \ldots 0 1)_b
\]
Die zweitkleinste Zahl ist $z_2=2z_1$. Wir wollen $x=\frac{z_1+z_2}{2}$ in Fixkommadarstellung darstellen, müssen wir entweder zu $z_1$ abrunden oder zu $z_2$ aufrunden.
\end{example}
\begin{fluff}
Der relative Fehler dieses Runden ist:
\[
\frac{|x-z_1|}{|x|}=\frac{1}{3}
\]
Analog für $z_2$. \\
Solche Fehler machen jede Rechnung unbrauchbar.
\end{fluff}