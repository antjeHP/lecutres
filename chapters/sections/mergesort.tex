\section{Mergesort}
Zu erst wollen wir folgendes beobachten:
\begin{lemma}
	Gegeben seien zwei sortiere Mengen
	\begin{itemize}
		\item $S_x=\{x_1<\ldots<x_m\}$
		\item $S_y=\{y_1<-\ldots<y_n\}$
	\end{itemize}
Dann lässt sich die Menge $S=S_x \cup S_y$ mit linearem Aufwand sortieren. Genauer werden $m+n-1$ Vergleiche benötigt.			
\end{lemma}
\begin{proof}
Konstruktiv, durch den entsprechenden Algorithmus.
\end{proof}
\begin{algorithm}[H]
\label{alg:merge}
\caption{Merge}
\KwData{Sortierte Mengen $S_x$ und $S_y$}
\KwResult{Sortierte Menge $S= S_x \cup S_y$}
Initialisiere $i=j=k=1$ \\
\While{$i\le m$ und $j\le n$}{
\If{$x_i < y_j$}{

$z_k=x_{i}$ \\
$i=i+1$
}
\Else{
$z_k=y_j$ \\
$j=j+1$

}
$k=k+1$
}
\For{$l \gets 0$ \KwTo $m-i$}{
$z_{k+l}=x_{k+l}$
}
\For{$l \gets 0$ \KwTo $n-j$}{
$z_{k+l}=y_{k+l}$
}
\end{algorithm}
Basierend auf dieser Beobachtung können wir eine divide-and-conquer Strategie angeben um Mengen der Länge $n=2^{m}, m \in  \N$ zu sortieren. \\
\begin{algorithm}[H]
	\label{alg:mergesort}
	\caption{Mergesort}
	\KwData{Menge $S= \{z_1,\ldots,z_n\}$}
	\KwResult{Sortierte Menge $S^{\pi}= \{z_{\pi_1} < \ldots < z_{\pi_n}\} $}
	\If{$n=1$}{
	$S^{\pi}=S$
	}
	\Else{
		\begin{itemize}
			\item Sortiere \\
	$L=\{z_1,\ldots, z_{\frac{n}{2}}\}$ \\
	$R= \{z_{\frac{n}{2}+1},\ldots, z_n\}$
	mittels Mergesort zu $L^{\pi}$ und $R^{\pi}$ \\
	\item Sortiere $L^{\pi} \cup R^{\pi}$ mittels Merge-Algorithmus \ref{alg:merge} zu $S^{\pi}$
	\end{itemize}
	}
\end{algorithm}
\begin{example}
Mergesort der Menge $\{20,7,84,31,71,42,18,10\} $
\end{example}
\begin{remark}
Da Mergesort sich selbst aufruft, sprechen wir von einem \emph{rekursiven Algorithmus}. Im Allgemeinem ist es schwierig zu beurteilen, ob solche Algorithmen terminieren. Im Fall von Mergesort ist der Fall jedoch klar, da die Rekursion im Fall $n=1$ abgebrochen wird.
\end{remark}
\begin{theorem}
	\label{thm:mergesort}
	Der Aufwand von Mergesort ist $\mathcal{O}(n \log n)$
\end{theorem}
\begin{proof}
Bezeichne A(n) den Aufwand für das Sortieren einer $n=2^{m}$ elementigen Menge mittels Mergesort. Dann gilt:
\begin{align*}
	A(1)&=0 \\
	A(n)&=n-1 + 2A(\frac{n}{2})
\end{align*}
Auflösen der Rekursion ergibt:
\begin{align*}
	A(n)
	&=n-1+2A(\frac{n}{2})\\
	&=2n-1-2+4A(\frac{n}{4})\\
	&=\ldots \\
	&=mn-\sum_{i=0}^{m-1}2^{i} \\
	&= mn- \frac{1-2^{m}}{1-2} \\
	&= (m-1)n+1
\end{align*}
$m=\log_2(n)=\frac{\log(n)}{\log(2)}$ impliziert die Behauptung
\end{proof}
$\log n \ll n$, also ist die Verbesserung von $\mathcal{O}(n^2)$ nach $\mathcal{O}(n\log(n)$ signifikant.
Man nennt das Wachstum auch beinahe linear.

\begin{remark}
Die Implementierung von Mergesort als in-place-Algorithmus ist je nach Datenstrukturen trickreich. Deshalb wird oft nicht in-place implementiert, weshalb für jeden "divide"-Schritt zusätzlicher Speicherplatz implementiert werden muss.
\end{remark}