\section{Zusammenhang}
\begin{definition}
Sei $G=(V,E)$ ein ungerichteter Graph $\emptyset \neq C \subset V$
\begin{itemize}
	\item Die Menge $C$ heißt \emph{zusammenhängend}, falls je zwei Knoten $v,w \in C , v\neq w,$ voneinander erreichbar sind, das heißt, $v \in post^*(w), w \in post^{*}(v)$. \\
Ist G ein Digraph, so heißt $C$ zusammenhängend, falls $C$ im zugrundeliegenden ungerichteten Graphen zusammenhängend ist.
\item Eine zusammenhängende Knotenmenge heißt \emph{Zusammenhangskomponente}, falls sie \emph{maximal} ist. Das bedeutet, es gibt keine weitere zusammenhängende Knotenmenge $D \subset V$ mit $C \subsetneq V$.
\item Ein Graph heißt zusammenhängend, falls $V$ zusammenhängend ist.
\end{itemize}
\end{definition}
\begin{example}
Dies ist ein unzusammenhängender Graph mit drei Zusammenhangskomponenten:
\begin{center}
\begin{tikzpicture}
    \node (A) at (0,0) {•};
    \node (B) at (0,-1) {•};
    \node (C) at (0,-2) {•};
    
    \node (D) at (2,0) {•};
    \node (E) at (1,-1) {•};

    \node (F) at (-1,0) {•};
    \node (G) at (-1,-1) {•};
    \node (H) at (-1,-2) {•};
    \node (I) at (-2,-1) {•};

    \path [-] (A) edge node {} (B);
    \path [-] (C) edge node {} (B);
    \path [-] (D) edge node {} (E);
    \path [-] (F) edge node {} (G);
    \path [-] (G) edge node {} (H);
    \path [-] (I) edge node {} (G);
\end{tikzpicture}
\end{center}
Analog hierzu ist der folgende Graph zusammenhängend:
\begin{center}
\begin{tikzpicture}
    \node (A) at (0,0) {•};
    \node (B) at (0,-1) {•};
    \node (C) at (0,-2) {•};

    \node (D) at (2,0) {•};
    \node (E) at (1,-1) {•};

    \node (F) at (-1,0) {•};
    \node (G) at (-1,-1) {•};
    \node (H) at (-1,-2) {•};
    \node (I) at (-2,-1) {•};

    \path [-] (A) edge node {} (B);
    \path [-] (C) edge node {} (B);
    \path [-] (D) edge node {} (E);
    \path [-] (F) edge node {} (G);
    \path [-] (G) edge node {} (H);
    \path [-] (I) edge node {} (G);
    
    \path [-] (B) edge node {} (G);
    \path [-] (B) edge node {} (E);
\end{tikzpicture}
\end{center}
\end{example}
Wir können beobachten, dass die Zusammenhangskomponenten eines ungerichteten Graphen die Äquivalenzklassen der Knotenmenge V unter der Äquivalenzrelation
\[
v=w \iff \{v\} \cup post^{*}(v) = \{w\} post^{*}(w) 
\]
ist. \\
Insbesondere zerfällt $G$ in paarweise disjunkte Zusammenhangskomponenten $C_1,\ldots, C_r$ mit
\begin{align*}
	V &= \bigcup_{i=1}^{r} C_i \\
	E &= \bigcup_{i=1}^{r}E_i
\end{align*}
wobei $E_i= E \cap \{X \subset C_i : |X|=2\}$.

\begin{theorem}
	\label{thm:zusammenhang}
	Sei $G=(V,E)$ ein ungerichteter Graph mit $n=|V| \ge 1$ Knoten und $m=|E|$ Kanten. Ist $G$ zusammenhängend, so folgt für die Anzahl Knoten und Kanten, dass 
\[
m \ge n-1
\]
\end{theorem}
\begin{proof}

\end{proof}
Per vollständiger Induktion:

\begin{itemize}[label=$\lozenge$, itemsep=2ex]
	\item IV \underline{n=1} \\ $m=0=n-1$
	\item IS \underline{$n \ge 3$} \\ Wähle $v \in  V$ mit:
		\[
		deg(v)= \min_{v \in V} deg(v) \eqqcolon k
		\]
		Da $G$ zusammenhängend ist, muss $k>0$ gelten. \\
		\underline{$k\ge 2$}
		\begin{align*}
		2m 
		&= 2|E| \\
		&= \sum_{w \in V} |deg{w} \\
		&\ge 2 |V| \\
		&= 2n 
		&\implies m\ge n\ge n-1
		\end{align*}
	\underline{$k=1$} \\ Sei $G'(V',E')$ derjenige Graph, der durch das Streichen von v und der zugehörigen Kante entsteht. $G'$ ist zusammenhängend, weil $G$ zusammenhängend ist. Die Induktionsannahme impliziert:
	\[
m-1=|E'| \ge |V'|-1 =(n-1)-1 = n-1 \implies m \ge n-1
	\]
\end{itemize}
