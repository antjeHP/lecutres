\section{Kondition linearer Gleichungssysteme}
\begin{recall}
Die Kondition beschreibt, wie sehr Fehler in den Eingangsdaten eines Problems
verstärkt oder gedämpft werden und sich auf einen Fehler der Ausgangsdaten übertragen.
\end{recall}
Zum Lösen von $Ax=b$, $A$ invertierbar, bezeichnen wir $\Delta b$ als einen Eingangsfehler in $b$. 
\begin{align*}
&\implies x+ \Delta x = A^{-1}(b+\Delta b) = A^{-1}b +A^{-1} \Delta b \\
&\implies \Delta x= A^{-1} \Delta b
\end{align*}
Für ein verträgliches Matrix-Vektorraum-Paar gilt dann: 
\[
\|\Delta x\|= \|A^{-1} \Delta b\| \le \|A^{-1}\|\|\Delta b\| 
\]
und 
\[
	\frac{\|\Delta x\|}{\|x\|} \le \|A^{-1}\| \frac{\|\Delta b\|}{\|x\|}\le  \|A^{-1}\| \|A\| \frac{\|\Delta b\|}{\|b\|}
\]
\begin{definition}
Der Faktor 
\[
cond_M (A) = \|A^{-1}\|_M \|A\|_M 
\]
wird als \emph{Kondition} der Matrix $A$ bezüglich der Matrixnorm $\|•\|_M$ bezeichnet. 
\end{definition}
\begin{example}
	\label{eg:lgs-kondition}
Betrachte 
\[
A= \begin{bmatrix}
	10^{-3} & 1 \\
	1 & 1
\end{bmatrix}
\]
\begin{align*}
	&\implies A^{-1} \approx \begin{bmatrix}
		-1.001 & 1.001 \\
		1.001 & -0.001
	\end{bmatrix} \\
	&\implies \|A\|_{\infty} =2 & \|A^{-1}\|_{\infty} \approx 2 \\
	&\implies cond_{\infty}(A) \approx 4
\end{align*}
A ist also bezüglich $\|•\|_{\infty}$ gut konditioniert.
\end{example}
