\section{Zahlensystem}
Die Darstellung von Zahlen basiert auf sogenannten \emph{Zahlensystemen}.
Diese Zahlensysteme unterscheiden sich in der Wahl des zugrundeliegendes Alphabets.
Unsere Zahl entspricht dann einem Wort, bestehend aus Elementen des Alphabets.\\


\begin{definition}[Alphabet]
Es sei $\N = \{ 1,2,3\ldots \}$ und $b \in  \N$. \\
Wir bezeichnen mit $\sum_{b}$ das Alphabet des \emph{b-adischen Zahlensystems}
\end{definition}

\begin{example}
Verschiedene Zahlensysteme
\begin{enumerate}
	\item Dezimalsystem: $\sum_{10}= \{0,1,\ldots,9\}$ \\ wie $(384)_10$
	\item Dual bzw. Binärsystem $\sum_2 = \{0,1\}$ \\ wie $(42)_10 = (101010)_2$
	\item \ldots
\end{enumerate}
\end{example}
Um die Zahlendarstellung sinnvoll zu nutzen Ergibt sich folgendes:
\begin{theorem}[]
Seien $b,n \in  \N, b>1 $. \\
Dann ist jede ganze, nicht-negative Zahl $z$ mit $0\le z \le b^{n}-1$ \emph{eindeutig} als Wort der Länge $n$ über $\sum_b$ darstellbar durch:
\[
z=\sum_{i=0}^{n-1}z_ib^{i}
\]
mit $z_i \in \sum_b$ für alle $i=0,1,\ldots,n-1$. \\
Wir schreiben vereinfachend: 
\[
z=(z_{n-1},\ldots, z_0)_b
\]
\end{theorem}
\begin{proof}

\begin{itemize}
\item Induktion \medskip
\begin{itemize}[label=$\lozenge$, itemsep=2ex]
\item \emph{Induktionsanfang}: \underline{$z<b$} hat die eindeutige Darstellung $z_0=z$ und $z_i=0$ sonst.



\item \emph{Induktionsschritt}: \underline{$z-1 \to z\ge b$} \\   

Betrachte
\[
z= \left\lfloor{\frac{z}{b}}\right\rfloor \cdot b + (z \mod b)
\]
Da $\hat{z}<z$ besitzt die eindeutige Darstellung
\[
\hat{z}= (\hat{z_{n-1}},\ldots, \hat{z_0})_b
\]
Bemerke $\hat{z_{n-1}}=0$, da 
\[
\left( \hat{z_{n-1}}b^{n-1} \right)b \le z \le b^{n}-1
\]
Also ist $z_b$ eine n-stellige Darstellung von z in b-adischen Zahlensystem

\end{itemize}

\item Eindeutigkeit wird durch Widerspruch gezeigt. \\
\emph{Angenommen:} Es gibt zwei verschiedene Darstellungen
\[
z=\left( z_{n-1}^{(2)},\ldots, z_0^{(2)} \right)_b = \left(z_{n-1}^{(1)},\ldots,z_0^{(1)} \right)_b
\]
Sei $m \in  \N$ der größte Index mit $z_m^{(1)} \neq z_m^{(2)}$, \\
Ohne Beschränkung der Allgemeinheit kann gesagt werden $z_m^{(1)>z_m^{(2)}}$. \\
Dann müssten die Stellen $0,1,\ldots,m-1$ den niedrigeren Wert von $z_m^{(2)}$ kompensieren. Die größte mit diesen Stellen darstellbare Zahl ist aber:
\[
\sum_{i=0}^{m-1}(b-1)b^{i}= (b-1)\sum_{i=0}^{m-1}b^{i}= b^{m}-1 \text{.}
\]
Da $b^{m}$ der kleinstmögliche Wert der fehlende Stelle m ist, kann diese aber nicht kompensiert werden. Dies ist ein Widerspruch.
\end{itemize}
\end{proof}%
Durch den Beweis ergibt sich sofort ein Algorithmus zur Umwandlung einer Zahl in ein anderes Zahlensystem:
\begin{example}
Umwandlung von $(1364)_{10}$ in das Oktalsystem.
\begin{itemize}
	\item $1364 = 170 \cdot 8 +4$
	\item $170 = 21 \cdot 8 +2$
	\item \ldots
	\item $0 \cdot 8 + 2$
\end{itemize}
$\implies (2524)_8$
\end{example}

\begin{algorithm}[H]
 \caption{Bestimmung der b-adischen Darstellung}
 \KwData{Dezimalzahl $z \in \N_0$, Basis $b \in \N$}
 \KwResult{b-adische Darstellung $(z_{n-1},\ldots, z_0)_b $}
 Initialisiere $i=0$ \\
 \While{$z>0$}{
  $z_i = z \mod b $\\
  $z= \left\lfloor \frac{z}{b} \right\rfloor$ \\
  $i=i+1$
 }
 \end{algorithm}

\paragraph{Beobachtung}
Wir sehen, dass das Honor-Schmea nur eine Schleife, Additionen und Multiplikationen benötigt. 
Diese Operationen können wir am Computer durchführen. 
Im Gegensatz dazu steht die Potenz $b^{i}$ in modernen Programmiersprachen zwar zur Verfügung, wird aber im Hintergrund oft auf Multiplikationen zurückgeführt.
Man überprüft leicht, dass das Honor-Schema weniger Multiplikationen benötigt und somit schnellt ist.
