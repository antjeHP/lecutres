\section{Binominal-Koeffizienten}
Grundlegend handelt es sich hierbei um eine "Funktion" um Grundaufgaben der Kombinatorik zu lösen. Er gibt an, auf wie viele verschiedene Arten k Objekte aus einer Menge von n verschiedenen Objekte ausgewählt werden können.

\begin{notation}
Wir sagen für ${n \choose k}$ "n über k" oder "n aus k"
\end{notation}
\begin{definition}
Wir schreiben:
\[
{n \choose k} = \frac{n+(n+1)\cdot \ldots \cdot n(k+1)}{k!}= \prod_{j=1}^{k} \frac{(n+1-j)}{j}  
\]
falls $k>n$ dann gilt: ${n \choose k}=0$, ebenfalls gilt: ${n \choose 0} = {n \choose n} =1$ und ${n \choose n-1} = n$ \\
Sollte $k\le n$ sein gilt:
\[
{n \choose k} = \frac{n!}{k! (n-k)!} = {n \choose n-k}
\]
\end{definition}
\begin{lemma}
	$\forall_{n \in \N_0, \forall_{k \in \Z} }$  gilt:
	\[
	{n+1 \choose k+1} = {n \choose k} + {n \choose k+1}
	\]
\end{lemma}
\begin{proof}
Für $k\le  0$ klar. \\
Sei $k\ge 0$ dann gilt:
\begin{align*}
	{n \choose k} + {n \choose k+1} &= \frac{n!}{k! \cdot (n-k)!}+ \frac{n!}{(k+1)!\cdot (n-(k+1))!} \\
					&= \frac{(n+1)!}{(k+1)! \cdot ((n+1)-(k+1))!}\\
					&= {n+1 \choose k+1}
\end{align*}
\end{proof}
\paragraph{Pascalsche Dreieck}
\begin{center}
\begin{tabular}{>{$n=}l<{$\hspace{12pt}}*{13}{c}}
0 &&&&&&&1&&&&&&\\
1 &&&&&&1&&1&&&&&\\
2 &&&&&1&&2&&1&&&&\\
3 &&&&1&&3&&3&&1&&&\\
4 &&&1&&4&&6&&4&&1&&\\
5 &&1&&5&&10&&10&&5&&1&\\
6 &1&&6&&15&&20&&15&&6&&1
\end{tabular}
\end{center}
\begin{theorem}[Binomischer Lehrsatz]
Für alle $x,y \in \R$ und für alle $n \in \N_0$ gilt:
\[
	(x+y)^{n}= \sum_{k=0}^{n}{n \choose k} \cdot x^{n-k}\cdot y^{k}
\]
\end{theorem}
\begin{proof}
durch Induktion:
\begin{itemize}[label=$\lozenge$, itemsep=2ex]
	\item \underline{IA $n=1$}:
		\[
			(x+y)^{1}= x+y
		\]
	\item \underline{IS $n \to +1$}:
		\begin{align*}
			(x+y)^{n+1}&= (x+y)^{n}\cdot (x+y) \\
				   &= \sum_{k=0}^{n}{n \choose k} x^{n+1-k}y^{k}+ \sum_{k=0}^{n} x^{n-k}y^{k+1} \\
			&= \sum_{k=0}^{n} {n \choose k} x^{n+1-k} y^{k} + \sum_{k=1}^{n+1}{n \choose k} x^{+1-k}y^{k} \\
			&= \sum_{k=1}^{n+1}\left( {n \choose k} + {n \choose k+1} \right) x^{n+1-k}y^{k}
		\end{align*}
\end{itemize}
\end{proof}
\begin{corollary}
	Es folgt direkt:
\begin{itemize}
	\item $\sum_{k=0}^{n}{n \choose k} = 2^{n}$
	\item $\sum_{k=0}^{n}(-1)^{k}{n \choose k} = 0$ 
\end{itemize}
\end{corollary}
\begin{theorem}
	Die Anzahl der Anordnungen von n verschiedenen Elementen ist $n!$. 
\end{theorem}
\begin{proof}
per Indutkion:
\begin{itemize}[label=$\lozenge$, itemsep=2ex]
\item \underline{IA $n+1$ } $a_1$, also eine Anordnung
\item \underline{IS $n \to n+1$}
	Sei $N(n)$ die Anzahl der Anordnungen von $n$ Elementen. \\
	\emph{Annahme:} $N(n)=n!$ \\
	Gegeben sei $n+1$ Elemente $\implies n+1$ Möglichkeiten für den ersten Platz. Jede dieser Möglichkeiten erlaubt $N(n)=n!$ Möglichkeiten zu Anfang.
	Daraus folgt es gibt insgesamt $n! (n+1) = (n+1)!$ Optionen. 
\end{itemize}
\end{proof}
\begin{theorem}
	Die Anzahl der k-elementigen Teilmengen einer n-elementigen Menge ist ${n \choose k}$. 
\end{theorem}
\begin{proof}
per Indutkion nach n.
\begin{itemize}[label=$\lozenge$, itemsep=2ex]
	\item \underline{IA $n=0$}
		\[
		{0 \choose 0} = 1
		\]
	\item \underline{IS $n \to n+1$} 
	Sei $M = \{a_1,\ldots,a_{n+1}\} $ 
	Für $0\le k \le n+1$ gilt: \\
	k-elementigen Teilmengen zerfallen in 2 Klassen:
	\begin{itemize}
		\item Teilmengen mit $a_{n+1}$
		\item Teilmengen ohne $a_{n+1}$ 
	\end{itemize}
	Für ersten gilt: Mengen $\{a_{n+1}\} \cup M_k$ mit beliebiger k-elementiger Teilmenge von $\{a_1,\ldots, a_n\} \implies \exists {n \choose k-1} $ Teilmengen.\\
Für zweitens gilt: beliebige k-elementige Teilmengen von $\{a_1,\ldots,a_n\} \implies {n \choose k}$ Teilmengen. \\
Also insgesamt:
\[
{n \choose k-1} + {n \choose k} = {n+1 \choose k}
\]
Teilmengen. Damit ist $A(n+1)$ bewiesen.
\end{itemize}
\end{proof}
\begin{definition}
Für eine Menge M ist die Potenzmenge $\mathfrak{P}(M)$  die Menge aller Teilmengen von M.
\end{definition}
\begin{example}
Für $M=\{a_1,a_2\} $ ist die Potenzmenge von M $\mathfrak{P}(M)= \{\O,\{a_1\} ,\{a_2\} ,\{a_1,a_2\} \} $.
\end{example}
\begin{theorem}
Die Potenzmenge $\mathfrak{P}(M)$ einer n-elementigen Menge $M$ hat $2^{n}$ Elemente
\end{theorem}
\begin{proof}
Es gibt ${n \choose k}$ Teilmengen mit k-Elementen. Also insgesamt:
\[
\sum_{k=0}^{n}{n \choose k} = 2^{n}
\]
Teilmengen.
\end{proof}

%Multinominalkoeffizient
