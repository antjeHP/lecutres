\section{Aussagenlogik}
\begin{definition}[Aussage]
	Eine \emph{Aussage} ist eine Behauptung, die eindeutig feststeht, so ist diese entweder \emph{wahr} oder \emph{falsch}
\end{definition}
\begin{definition}
Falls A und B Aussagen gilt:
\begin{table}[htpb]
	\centering
	\label{tab:junktoren}
	\begin{tabular}{c|c}
		Symbol & Gesprochen \\
		\hline
		$\neg A$ & "nicht A" \\
		$A \vee B$ & "A oder B" \\
		$A \wedge B$ & "A und B" \\
		$A \implies B$ & "aus A folgt B", "A impliziert B" \\
		$A \iff B$ & "A ist äquivalent zu B", "A genau dann wenn B"
	\end{tabular}
\end{table}
Die Wahrheitstafeln dazu sehen wie folgt aus:
\begin{table}[htpb]
	\centering
	\label{tab:wahrheitstafel}
	\begin{tabular}{c|c|c|c|c|c|c}
		$A$ & $B$ & $\neg A$ & $A \vee B$ & $A \wedge B$ & $A \implies B$ & $A \iff B$ \\ 
\hline
		w & w & f & w & w & w & w \\
		w & f & f & w & f & f & f \\
		f & w & w & w & f & w & f \\
		f & f & w & f & f & w & w 
	\end{tabular}
\end{table}
\end{definition}
\subsection{Gesetze der Aussagenlogik}
\begin{align*}
	(A \implies B) &\iff (\neg B \implies \neg A) \\
	(A \implies B) &\iff (\neg A \vee B) \\
	(A \iff B) &\iff (\neg A \iff \neg B) \\
	(A \wedge (A \implies B)) &\implies B \\
	((A\implies B) \wedge (\neg B)) &\implies \neg A \\
	((A \implies B) \wedge (B \implies C)) &\implies (A \implies C) \\
	(A) &\implies (A \vee B)\\
	(\neg A) &\implies (A \implies B)
\end{align*}
Wir können diese Gesetze durch Wahrheitstafeln beweisen. Das wichtigste Gesetzt ist die Kontraposition
\begin{theorem}[Kontraposition]
	\label{thm:kontraposition}
Falls $A$ und $B$ Aussagen gilt:
\[
	(A \implies B) \iff (\neg B \implies \neg A)
\]
\end{theorem}
\begin{remark}
Der Satz \ref{thm:kontraposition} wird oft benutzt um etwas zu beweisen. 
\end{remark}
Noch ein kurzer Beweis zu 
\[
	(A \implies B) \iff (\neg A \vee B)
\]
\begin{proof}
Durch eine Wahrheitstafel.
\begin{table}[htpb]
	\centering
	\begin{tabular}{c|c|c|c}
		$A$ & $B$ & $A \implies B$ & $\neg A \vee B$ \\
		\hline
		w & w & w & w \\
		w & f & f & f \\
		f & w & w & w \\
		f & f & w & w
	\end{tabular}
\end{table}
\end{proof}
