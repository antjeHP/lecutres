\section{Prädikatenlogik}
Aussagen, die von freien Variablen abhängen.
\begin{example}
	Zwei Ausssagenformen
\begin{itemize}
	\item Es gibt eine reele Zahl $x$ mit $x^2+1=0$
	\item Für alle natürlichen Zahlen $x$ gilt: $x^2-1=(x-1)(x+1)$ 
\end{itemize}
\end{example}
\begin{notation}[Quantoren]
	Wir verwenden die  Kurzformen
	\begin{itemize}
		\item $\exists$ für "es gibt ein"
		\item $\forall$ für "für alle" 
	\end{itemize}
\end{notation}
\paragraph{Regeln:}
\begin{align*}
	\neg(\forall_{x}: A(x)) &\iff (\exists_{x}: \neg A(x)) \\
	\neg (\exists_{x}:A(x) &\iff (\forall_{x}: \neg A(x))  \\
	\neg (\forall_{x}\forall_{y} : A(x,y)) &\iff (\forall_{x}\exists_{y}: \neg A(x,y))    
\end{align*}
\begin{warning}
Es gilt \underline{nicht}:
\[
\forall_{x \in M}: A(x) \implies \exists_{x \in M}:A(x)  
\]
Da $M$ z.B eine leere Menge sein kann.
\end{warning}
