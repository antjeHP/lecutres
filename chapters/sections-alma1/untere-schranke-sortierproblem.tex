\section{Untere Schranke für das Sortierproblem}
\paragraph{Frage:} Gibt es einen Sortieralgorithmus, der schneller ist als $\mathcal{O}(n\log(n))$?

\begin{theorem}
	\label{thm:sortierverfahren}
Jedes deterministische Sortierverfahren, das auf paarweisen Vergleichen basiert und keine Vorkenntnisse über die zu sortierende Menge hat, benötigt im schlimmsten Fall mindestens $\log_2(n!)$ Vergleiche zum Sortieren von $n$ verschiedenen Zahlen.
\end{theorem}
Bevor wir den Beweis machen gibt es noch einige Hinweise, die wir für den Beweis benötigen.
\begin{remark}
	Es gilt:
	\begin{itemize}
		\item  $\log_2(n!) = \Theta(n\log(n)$
		\item $\log_2(n) \le \log_2(n^{n}) = n\log_2(n) = \frac{1}{\log(2)}n \log n$
		\item $\log(n!) \ge \log(\frac{n}{2}\ldots \frac{n}{2}) \ge \log((\frac{n}{2})^{\frac{n}{2}}= \frac{n}{2}\log(\frac{n}{2}) = \frac{n}{2}\log(n) - n\log(2)$
	\end{itemize}
\end{remark}
Zum Beweis benötigen wir einen Entscheidungsbaum. Dieser kann jedem Sortieralgorithmus mit paarweisen Vergleichen zugeordnet werden, und illustriert das Verhalten des Algorithmus:
\begin{itemize}
	\item Innere Knoten des Entscheidungsbaums sind Vergleiche im Algorithmus.
	\item Ein Weg von der Wurzel zu einem Blatt entspricht der zeitlichen Abfolge der Vergleiche. Ein Weitergeben nach rechts illustriert einen richtigen Vergleich, nach links einen falschen.
	\item Die $n!$ Blätter des Baumes stellen die Permutationen der zu sortierenden Menge dar, die jeweils zur Abfolge der Vergleiche gehört.
\end{itemize}
\begin{example}
Baum für $\{z_1,z_2,z_3\}$: \\
%\begin{center}
%\begin{tikzpicture}
%
%
%
%
%
%\end{tikzpicture}
%\end{center}
\end{example}
\begin{proof}
\ref{thm:sortierverfahren} \\
\underline{Idee:} Finde eine untere Schranke für die Tiefe des Baumes. Dies ist die Mindestanzahl der im schlimmsten Fall mötigen Vergleiche.
Im besten Fall ist ein Baum ausbalanciert, d.h. alle Blätter haben die gleiche Tiefe $m$ und wir haben $2^{m}$ Blätter. Da wir $n!$ Permutationen haben (also Blätter), muss $2^{m}=n!$ gelten. Es folgt $m \log_2(n!)$
\end{proof}
Satz \ref{thm:sortierverfahren} sagt, dass Mergesort in der asymptotischen Laufzeit optimal ist. 
\paragraph{Frage:} Können wir etwas Ähnliches für Quicksort zeigen?

\begin{theorem}Alle Permutationen der Zahlen $z_1,\ldots,z_n$ seien gleich wahrscheinlich. Dann benötigen Sortierverfahren wie in Satz \ref{thm:sortierverfahren} im Mittel $\log_2(n!)$ Vergleiche.
\end{theorem}
\begin{proof}
Die mittlere Höhe $\overline{H}(z)$ des Entscheidungsbaums $\tau$ ist gegeben als der Mittelwert der Tiefen $t_{\tau}$ aller seiner Blätter $v$.
Sei $B(\tau)$ die Menge der Blätter und $\beta(\tau) = |B(\tau)|$ Dann gilt:
\[
\overline{H}(\overline{c})=\frac{1}{\beta(\overline{c})}\sum_{y=B(\overline{c}} t\overline{c}(v)
\]
Zu zeigen ist, dass:
\[
\overline{H}(\overline{c}) \ge \log_2(\beta(\overline{c}))
\]
Vollständige Induktion über die Höhe des Entscheidungsbaums $\overline{c}$.
\begin{itemize}[label=$\lozenge$, itemsep=2ex]
	\item Induktionsverankerung:$H(\overline{c})=0$ \\
		$\overline{c}$ besteht nur aus seiner Wurzel. $\implies \overline{H}(c)= \frac{1}{1}\sum_{y \in B(\tau)} t_{\tau}(v)=0= \log_2(1)$
	\item Induktionsschritt: $H(\overline{c})\le h-1 \implies H(\overline{c})=h$ \\
		Seien $\tau_l , \tau_r$ die Wurzeln des linken bzw. rechten Teilbaums zum Wurzelknoten von c:

	\begin{center}
	\begin{tikzpicture}
		\node(c) at (0,0) {c};
		\node(l) at (-1,-1) {$\tau_l$};
		\node(r) at (1,-1) {$\tau_r$};
		\node(1) at (-1.5,-2) {};
		\node(2) at (-0.5,-2) {};

		\path [-](c) edge node {} (l);
		\path [-](c) edge node {} (r);
		\path [-](l) edge node {} (1);
		\path [-](l) edge node {} (2);
		\path [-](1) edge node {} (2);
	\end{tikzpicture}
	\end{center}
\paragraph{Beobachte:}
\begin{itemize}
	\item $H(\tau_l), H(\tau_r) <h$
	\item $B(\tau)= B(\tau_l) \cup B(\tau_r)$
	\item $B(\tau_l) \cap B(\tau_r) = \emptyset$
	\item $\beta(\overline{c})= \beta(\tau_l) + \beta(\tau_r)$
	\item $\beta(\tau_l), \beta(\tau_r) \ge 1$
	\item $t_{\tau}(v)= t_{\tau_l}(v)+1$
	\item $t_{\tau}(v)= t_{\tau_r}(v)+1$
\end{itemize}
Rechne:
\begin{align*}
\overline{H}(\tau) 
&= \frac{1}{\beta(\tau)}\sum_{v \in B(\tau)} t_{\tau}(v) \\
&= \ldots \\
&\ge 1+\frac{1}{b} \min_{x \in [0,b]} (x \log_2(x) + (b-x) \log_2(b-x))
\end{align*}
Da
\begin{align*}
f'(x)
&= \log_2(x) -\log_2(b-x) \\
&= \log_2(\frac{x}{b-x}) &=0
\end{align*}
genau dann, wenn $x=\frac{b}{2}$, ist dies die einzige Möglichkeit, wo das Minimum in $(0,b)$ angenommen werden kann. Die anderen Möglichkeiten sind die Intervallrandpunkte, was ausgeschlossen ist.
Also gilt:
\begin{align*}
	\overline{H}(\tau) 
	&\ge 1+ \log_2(\frac{b}{2}) \\
	&= 1+ \log_2(\beta(\tau)) -1 \\
	&= \log_2(\beta(\tau))
\end{align*}
\end{itemize}
\end{proof}
