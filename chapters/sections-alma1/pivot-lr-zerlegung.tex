\section{LR-Zerlegung mit Spaltenpivotisierung}
\begin{example}
	Fortsetzung von Beispiel \ref{eg:lgs-kondition}
	\begin{align*}
	A= \begin{bmatrix}
		10^{-3} & 1 \\
		1 & 1
	\end{bmatrix}
	&, b= \begin{bmatrix}
	1 / 2
	\end{bmatrix}
	&\implies x = \begin{bmatrix}
	1.001001 \\ 0.998999
	\end{bmatrix}
	\end{align*}
	Löse in dreistelliger Arithmetik, mit kleiner Störung in der rechten Seite:
	\begin{align*}
	\begin{gmatrix}[b]
		0.001 & 1 & 1.01\\
		1 & 1 & 2.01
\rowops
\mult{0}{-1000}
\add{0}{1}
\end{gmatrix}
&\to \begin{bmatrix}[cc|c]
	0.001 & 1 & 1.01 \\
	0 & -999 & -1010
\end{bmatrix}
\\ 
&\implies \begin{cases}
	x_2= -1010 \boxslash (-999) = 1.01 \\
	x_1= (1.01 \boxminus (1 \boxcircle 1.01)) \boxslash 0.001 = 0
\end{cases}
	\end{align*}
Der Algorithmus muss also instabil sein.
\end{example}
\paragraph{Grund:} Das Inverse des Pivot-Elemts (1000) ist sehr groß und verstärkt den Fehler

\begin{definition}
	Sei $1\le i < j \le n$
	\begin{align*}
	T^{\{i,j\}} = \begin{bmatrix}
		1 \\
		& \ddots\\
		& & 1 \\
		& & & 0 & & & & 1\\
		& & & & 1 \\
		& & & & & \ddots\\
		& & & & & & 1\\
		& & & 1 & & & & 0 & \\
		& & & & & & & & 1 \\
		& & & & & & & & & \ddots \\
		& & & & & & & & & &  1
	\end{bmatrix}
	\end{align*}
	heißt Transpositionsmatrix.\\
	Eine Permutationsmatrix ist das Produkt mehrerer Transpositionsmatrizen.
	\end{definition}

\begin{lemma}
	Sei $T^{\{i,j\} }$ eine Transpositionsmatrix. Es gilt:
	\begin{enumerate}
		\item Multiplikation von links mit $T^{\{i,j\} }$ vertauscht i-te und j-te Zeile.
		\item Multiplikation von rechts mit $T^{\{i,j\} }$ vertauscht i-te und j-te Spalte. 
		\item $T^{\{i,j\} }$  ist symmetrisch. 
		\item $\left( T^{\{i,j\} } \right)^2=I$ 
	\end{enumerate}
	Ist $P$ eine Permutationsmatrix, so gilt: $P^{T}P=I$ 
\end{lemma}
\begin{proof}
a-d sind elementar. Letzter Punkt: Übung.
\end{proof}
\paragraph{Umseztung:} Anstatt im i-ten Schritt des Gauss-Algorithmus 
\[
A_i \to L_iA_i =A_{i+1}
\]
zu berechnen, berechnen wir:
\begin{equation}
\label{eqn:lr-gauss}
A_i \to  L_iP_iA_i = A_{i+1}
\end{equation}
\begin{lemma}
	Sei $1 \le i < j \le n$ und $P_i = T^{\{i,k\} }$ mit $i<k\le n$. Dann gilt: $P_iL_j=\tilde{L_j}P_i$, wobei $\tilde{L_j}$ und $L_j$ bis auf ein Vertauschen der $\tau_k^{(j)}$ und $\tau_i^{(j)}$ gleich sind.  
\end{lemma}
\begin{proof}
Durch nachrechnen. Also:
\[
\tilde{L_j} = P_iL_jPi
\]
\end{proof}
\begin{theorem}
	Sei A nicht singulär. Dann bestimmt der Gauss-Algorithmus mit Spaltenpivotisierung eine Zerlegung der Matrix $PA=\tilde{L}R$ , wobei $R$ die obere Dreiecksmatrix $A_n$ ,$P=P_{n-1} \ldots P_1$ eine Permutationsmatrix und $\tilde{L} = \tilde{L_1^{-1}}\ldots\tilde{L_{n-1}^{-1}}$ eine untere Dreiecksmatrix ist mit:
	\begin{align*}
		&\tilde{L_{n-1}} = L_{n-1} \\
		&\tilde{L_{n-2}} = P_{n-1} L_{n-2} P_{n-1} \\
		&\ldots  \ldots\\
		&\tilde{L_1}= P_{n-1}P_{n-2}\ldots P_2P_1 L_1 P_2 \ldots P_{n-2} P_{n-1}
	\end{align*}
\end{theorem}
\begin{proof}
Falls der Algorithmus nicht zusammenbricht:
\begin{align*}
	R = A_n &= L_{n-1} P_{n-1} A_{n-1} \\
		&= \tilde{L_{n-1}} P_{n-1} L_{n-2} P_{n-2} A_{n-2} \\
		&= \tilde{L_{n-1}}\tilde{L_{n-2}} P_{n-1} P_{n-2} L_{n-3} P_{n-3} A_{n-3} \\
		&\ldots \\
		&= \tilde{L_{n-1}}\ldots \tilde{L_1} P_{n-1}\ldots P_1 A \\
		&= \tilde{L^{-1}} \cdot P
\end{align*}
Sollte der Algorithmus abbrechen, folgt, dass ein Pivot-Element null wird. Daraus folgt, dass die Determinante von der Restmatrix null ist. Demnach ist auch die Determinante von der Matrix $A$ null. Das ist ein Widerspruch dazu, dass A invertierbar ist. 
\end{proof}
\begin{example}
Wir betrachten die $PA = \tilde{L}R$ Zerlegung.
\begin{align*}
&\begin{gmatrix}[b]
	1 & 1 & 0 & 2 \\
	\frac{1}{2} & \frac{1}{2} & 2 & -1 \\
	-1 & 0 & -\frac{1}{8} & -5 \\
	1 & -7 & 9 & 10 
\rowops
\mult{0}{-\frac{1}{2}}
\add{0}{1}
\add{0}{2}
\mult{0}{-1}
\add{0}{3}
\end{gmatrix}
&\to \begin{gmatrix}[b]
	1&1&0&2\\
	0 & 0 & 2 & -2 \\
	0 & 1 & -\frac{1}{8} & -3 \\
	0 & -8 & 9 & 8 
\rowops
\swap{1}{3}
\end{gmatrix}
	\overset{P_2}{\to} \\ &\begin{gmatrix}[b]
	1 & 1 & 0 & 2 \\
	0 & -8 & 9 & 8 \\
	0 & 1 & -\frac{1}{8} &-3 \\
	0 & 0 & 2 & -2
\rowops
\mult{1}{\frac{1}{8}}
\add{1}{2}
\end{gmatrix} &\to \begin{gmatrix}[b]
1 & 1 & 0 & 2 \\
0 & -8 & 9 & 8 \\
0 & 0 & 1 & -2 \\
0 & 0 & 2 & -2
\rowops
\swap{2}{3}
\end{gmatrix} \overset{P_3}{\to} \\
			      &\begin{gmatrix}[b]
	1 & 1 & 0&2 \\
	0 & -8 & 9 & 8 \\
	0 & 0 & 2 &-2 \\
	0 & 0 & 1 & -2
\rowops
\mult{2}{-\frac{1}{2}}
\add{2}{3}
\end{gmatrix} &\to \begin{bmatrix}
1 & 1 &0 &2 \\
0&-8&9&8\\
0 & 0 & 2 &-2 \\
0 & 0 & 0 &-1
\end{bmatrix} = R
\end{align*}
\begin{align*}
PA= \begin{bmatrix}
	1 & 1 & 0 &2 \\
	1 & -7 & 9 & 10 \\
	\frac{1}{2} & \frac{1}{2} & 2 & -1 \\
	-1 & 0 & -\frac{1}{8} & -5
\end{bmatrix} & L= \begin{bmatrix}
1 \\
\frac{1}{2} & 1 \\
-1 & -\frac{1}{8} &1 \\
1 & 0& \frac{1}{2}& 1
\end{bmatrix} \to \tilde{L} = \begin{bmatrix}
1 \\
1 & 1 \\
\frac{1}{2} & 0 & 1 \\
-1 & -\frac{1}{8} & \frac{1}{2} &1
\end{bmatrix}
\end{align*}
\end{example}
\begin{remark}
Um $\tilde{L}$ zu bekommen, berechne zuerst $L$, und wende dann die Transpositionsmatrizen $P_j$  auf die i-te Spalte an.
\end{remark}
\begin{example}
Betrachte
\begin{align*}
A= \begin{bmatrix}
	10^{-3} & 1 \\ 1 & 1
\end{bmatrix},
b= \begin{bmatrix}
1 \\ 2 
\end{bmatrix} &\implies x= \begin{bmatrix}
1.001001 \\ 0.998999
\end{bmatrix}
\end{align*}
in 3-stelliger Arithmetik wie zuvor.
\begin{align*}
\begin{bmatrix}[c c | c]
	0.001 & 1 & 1.01 \\
	1 & 1 & 2.01
\end{bmatrix} \to \begin{bmatrix}[c c | c]
	1 & 1 & 2.01 \\
	0.001 & 1 & 1.01 
\end{bmatrix} \to \begin{bmatrix}[c c | c]
	1 & 1 &2.01 \\
	0 & -0.999 & 1.01
\end{bmatrix} \implies \begin{cases}
	x_2= 1.01 \boxslash (-0.999) = 1.01 \\
	x_1= (2.01 \boxminus 1.01) =1
\end{cases}
\end{align*}
\end{example}
\begin{remark}
Die Spaltenpivotisierung reicht in der Praxis meistens aus. Eine stabilere Idee ist die totale Pivotisierung. Hierbei werden Zeilen und Spalten so permutiert, dass das betragsmäßig größte Element von $A_i$ im i-ten Schritt das Pivot-Element ist. Der Aufwand ist jedoch nicht mehr vernachlässigbar. 
\end{remark}
