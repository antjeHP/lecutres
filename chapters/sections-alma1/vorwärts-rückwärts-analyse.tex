\section{Vorwärts- und Rückwertsanalyse}
Abstrakt gesehen entspricht das Lösen eines Problems dem Auswerten einer Funktion $f$.
Beim numerischen Auswerten von $f$ können verschiedene Fehler passieren:
%Hier muss noch das Diagramm von Vorlesung 5 hin.

\begin{definition}[Fehlerarten]
Sei $D \subset \R$ eine Menge von Eingangsdaten und $W \subset \R$ die Menge der möglichen Ergebnisse. Wir unterscheiden folgende Fehlerarten bei der Auswertung von $f \colon D \to W$.
\begin{itemize}
\item \underline{Datenfehler} Typischerweise sind die Eingangsdaten $x \in D$ nicht exakt,
sondern mit einem Datenfehler $\Delta x$ behaftet. Die gestörten Eingangsdaten $\tilde{x}=x+ \Delta x$ produzieren einen Fehler $f(x+\Delta x)-f(x)$.
\item \underline{Verfahrensfehler} Exakte Verfahren enden bei exakter Rechnung nach endlich vielen Operationen mit dem exakten Ergebnis.
Näherungsverfahren enden in Abhängigkeit bestimmter Kriterien mit einer Näherung $\tilde{y}$ für die Lösung $y \in W$.
\item \underline{Rundungsfehler} Verursacht durch Maschinenzahlen und Rundungen während der Arithmetik.
\end{itemize}
\end{definition}
Ein Teilgebiet der Numerik versucht diese Fehler durch eine Fehleranalyse zu quantifizieren.
Hierzu verfolgt man die Auswirkungen von allen Fehlern, die in den einzelnen Schritten vorkommen können.
\begin{definition}[Analysearten]
Bei der \emph{Vorwärtsanalyse} wird der Fehler von Schritt zu Schritt verfolgt und der akkumulierte Fehler für jedes Teil-Ergebnis abgeschätzt. \\
Bei der \emph{Rückwertsanalyse} geschieht die Verfolgung des Fehlers hingegen so, dass jedes Zwischenergebnis als exakt berechnetes Ergebnis zu gestörten Daten interpretiert wird, d.h , der akkumulierte Fehler wird als Datenfehler interpretiert.
\end{definition}
\begin{example}
Betrachte $f(x,y)=x+y$ \\
\begin{itemize}
	\item Vorwärtsanalyse: $\boxed{f} = x \boxplus y= (x+y)(1+\varepsilon)$
	\item Rückwärtsanalyse: $x \boxplus y= x(1+\varepsilon)+y(1+\varepsilon)= f(x(1+\varepsilon), y(1+\varepsilon)$
\end{itemize}
mit $|\varepsilon|\le \varepsilon_{mach}$
\end{example}
In der Praxis ist die Vorwärtsanalyse kaum durchführbar. 
Für die meisten Algorithmen ist, wenn überhaupt, nur eine Rückwärtsanalyse bekannt.
\begin{example}
\label{eg:rückwärtsanalyse}
Fortsetzung von Beispiel \ref{eg:pq} \\
Führe Rückwärtsanalyse durch zu Algorithmus \ref{alg:nst1} für die Betrags-kleinere Nullstelle $x_2$ durch:
Dann ist
\[
f(p,q)=p-\sqrt[2]{p^2-q}
\]
mit $p>0$. 
Für $|\varepsilon_i|\le \varepsilon_{mach}, i=1,2,3,4$ betrachten wir:
\begin{align*}
\left( p-\sqrt[2]{(p^2(1+\varepsilon_1)-q)(1+\varepsilon_2)} (1+\varepsilon_3) \right)(1+\varepsilon_4)
&= p(1+\varepsilon_4)-\sqrt[2]{(p^2(1+\varepsilon_1)-q)(1+\varepsilon_2)(1+\varepsilon_3)^2(1+\varepsilon_4)^2}\\
&= p^2(1+\varepsilon_1)(1+\varepsilon_3)^2(1+\varepsilon_4)^2-q(1+\varepsilon_2)(1+\varepsilon_3)^2(1+\varepsilon_4)^2 \\
&= \ldots \\
&= p(1+\varepsilon_4)-\sqrt[2]{p^2(1+\varepsilon_4)^2-q(1+\varepsilon_7)} \\
&= f(p(1+\varepsilon_4),q(1+\varepsilon_7))
\end{align*}
Die Abschätzung für $\varepsilon_7$ explodiert, falls $0<|q| \ll 1 < p$. Dies war in Beispiel \ref{eg:pq} der Fall.
\end{example}