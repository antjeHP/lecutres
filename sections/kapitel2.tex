\section{Fehleranalyse}
\subsection{Rechnerarithmetik}
\begin{example}
Betrachte $F=F(10,5,-4,5)$ und Maschinenzahlen
\begin{align*}
x=2.5684 \cdot 10^{0}= 2.56840000 \\
y=3.2791 \cdot 10^{-3} = 0.0032791
\end{align*}
Es gilt:\\
\begin{center}
$\begin{rcases}
	x+y= 2.5716792 \\
	x-y= 2.5651209 \\
	x \cdot y= 0.00842204044 \\
	\frac{x}{y}= 783.2637004
\end{rcases} \not\in F$
\end{center}
\end{example}
\begin{remark}
Die Menge $F(b,t,e_{min},e_{max})$ ist nicht abgeschlossen bezüglich der Grundrechenarten und können somit im Allgemeinen nicht im Computer implementiert werden.
\end{remark}
\paragraph{Lösung}
Wir runden das Ergebnis und implementieren so eine Pseudoarithmetik.
Das bedeutet wir ersetzen $o \in \{+,-,\cdot,/\} $ durch \boxed{o}, definiert durch x \boxed{o} y $\coloneqq rd(x \circ )y$. 
%Dieser Abschnitt muss noch verbessert werden dringlichst
\begin{equation}\label{eqn:rundung}
s=t
\end{equation}
Auf Hardwareebene wird üblicherweise mit einer längeren Matisse gearbeitet und dann normalisiert und gerundet. Dies entspricht dem IEEE 754 Standard.
\begin{remark}
	Für $|x|,|y|, |x \circ y| \in [z_{min},z_{max}]$ impliziert \eqref{eqn:rundung}, dass
\end{remark}
